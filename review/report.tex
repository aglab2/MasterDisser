\documentclass[12pt, a4paper]{article}

\usepackage[top=0.5in, bottom=0.5in, left=1.2in, right=1.2in]{geometry}
\usepackage{setspace}
\usepackage[utf8]{inputenc}
\usepackage[russian]{babel}
%\usepackage[none]{hyphenat}

\pagenumbering{gobble}

\newcommand{\signature}[2][20em]{%
  \hspace{-8em}
  \begin{tabular}[t]{ p{#1} p{#1} }
    \raisebox{-.5ex}[0pt][0pt]{\bfseries } & \\
    \cline{2-2}
    & \centering\scriptsize\itshape (#2)
  \end{tabular}
}

\begin{document}
\setlength{\parskip}{1.3em}
\begin{center}
\textbf{ОТЗЫВ НАУЧНОГО РУКОВОДИТЕЛЯ} \\
\textbf{на выпускную квалификационную работу} \\
\textbf{студента Д.В. Копырина "Перехват файловых операций для создания виртуальных sparse-файлов в операционной системе macOS"}
\end{center}

\setlength{\parskip}{0.5em}
Дипломная работа представляет метод перехвата файловых операций для создания виртуальных sparse-файлов, которые могут быть использованы для обмена информацией между пользователями по сети. Исследование является актуальным, так как технологии синхронизации данных являются неотъемлемой частью бэкап-решений и предоставляет дополнительный уровень защиты файлов.

Копырин Д.В. рассматривает устройство подсистемы Virtual File System и описывает способ создания sparse файлов, при чтении которых происходит подкачка данных с удаленного хранилища. В работе производится исследование различных вариантов перехватов и выбирается наилучший способ.

При реализации расширения ядра было рассмотрено множество краевых случаев как кэширование pagein операций в UBC, что имеет практическую ценность. Алгоритмы организации синхронного транспорта и управления блоками файла являются теоретически значимыми.

Положительной стороной работы является всестороннее описание действий над файлами и их отображение в таблице $vnode$ операций, рассмотрены способы противодействия нежелательному чтению сканирующими процессами. Недостатком работы стал уклон в практическую часть, представлены значительные части кода реализации.

При работе над дипломом, студент проявил аналитические способности: была изучена существенная часть кода ядра XNU, прочитаны книги по внутреннему устройству операционной системы. Во время обучения, студент выступал на конференции МФТИ, где были представлены смежные к диплому темы. В настоящее время Копырин Д.В. является эрудированным молодым специалистом, умеющим работать в коллективе и самостоятельно ставить задачи и их решать. 

Дипломная работа соответствует требованиям положения о ВКР и может быть рекомендована к защите с оценкой "отлично". Автор достоин присуждения степени "магистр" по направлению подготовки 03.04.01 "Прикладная математика и физика".

\setlength{\parskip}{1.2em}
\hfill \textbf{Научный руководитель:} \hspace{8.5em}

\signature{ФИО}

\signature{ученая степень, ученое звание, должность}

\signature{подпись научного руководителя}

\hfill "\rule{20pt}{1pt}" \rule{90pt}{1pt} 2019 г.

\end{document}