\documentclass[12pt, a4paper]{article}

\usepackage[top=0.8in, bottom=1in, left=1.2in, right=1.2in]{geometry}
\usepackage{setspace}
\usepackage[utf8]{inputenc}
\usepackage[russian]{babel}
%\usepackage[none]{hyphenat}

\pagenumbering{gobble}

\newcommand{\signature}[2][20em]{%
  \hspace{-8em}
  \begin{tabular}[t]{ p{#1} p{#1} }
    \raisebox{-.5ex}[0pt][0pt]{\bfseries } & \\
    \cline{2-2}
    & \centering\scriptsize\itshape (#2)
  \end{tabular}
}

\begin{document}
\setlength{\parskip}{1.3em}
\begin{center}
\textbf{РЕЦЕНЗИЯ} \\
\textbf{на выпускную квалификационную работу} \\
\textbf{студента Д.В. Копырина "Перехват файловый операций для создания виртуальных sparse-файлов в операционной системе macOS"}
\end{center}

\setlength{\parskip}{0.5em}
Работа Копырина Дениса посвящена перехвату файловых операций для подмены содержимого файла с целью обеспечения его удаленного хранения и получения изменений локального содержимого по частям. В настоящее время облачные технологии являются важной частью IT-индустрии, вопросы синхронизация файлов между устройствами и стриминг являются актуальными.

Автор работы подробно изучает структуры данных ядра, отвечающие за представление файлов в операционной системе. На этой основе выбирается наиболее безопасный и производительный способ перехватов операций чтения и записи. Также был рассмотрен метод подмены объектов виртуальной памяти, который является эффективным и элегантным решением проблемы кэширования невалидных пустых страниц.

В работе рассмотрен важный теоретический аспект реализации - оптимальный алгоритм управления регионами виртуального файла. Были установлены свойства валидности структуры данных, доказана корректность алгоритма и приведена имплементация.

Результатом диплома стала разработка расширение ядра macOS и mock клиента. Решение имеет практическую ценность, виртуальные файлы являются прозрачными для пользователя, нативно поддерживается POSIX API.

Достоинством диплома является глубокое исследование работы файлов и в том числе memory-mapped файлов, показан вариант защиты от нежелательного чтения-подкачки.

К недостаткам работы можно отнести излишне подробное описание структур ядра, некоторые представленные примеры реализации являются слишком громоздкими.

Дипломная работа соответствует требованиям ГОСТа и может быть рекомендована к защите с оценкой "отлично". Автор достоин присуждения степени "магистр" по направлению подготовки 03.04.01 "Прикладная математика и физика".

\setlength{\parskip}{1.2em}
\hfill \textbf{Рецензент:} \hspace{15.7em}

\signature{ФИО}

\signature{ученая степень, ученое звание, должность}

\signature{подпись рецензента}

\hfill "\rule{20pt}{1pt}" \rule{90pt}{1pt} 2019 г.

\end{document}